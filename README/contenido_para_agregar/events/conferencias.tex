

\begin{itemize}

\item[ ]	% MDW2022
Institut für Volksmusikforschung und Ethnomusikologie, Universität für Musik und darstellende Kunst Wien.
``Musical Traits and Performance Practice of Uruguayan Candombe Drumming:
A Computational Musicological Approach''.
Viena, Austria, 19 de octubre, 2022.\\
Luis Jure.

\item[ ]	% MaxPlanck2022
Max Planck Institute for Empirical Aesthetics.
``A computational approach to Uruguayan Candombe drumming''.
Fránfort del Meno, Alemania, 10 de agosto, 2022.\\
Luis Jure, Martín Rocamora.

\item[ ]	% stanford2018
Stanford University, Center for Computer Research in Music and Acoustics -- CCRMA.
``Musical Traits and Performance Practice of Uruguayan Candombe Drumming:
A Computational Musicological Approach''.
Palo Alto, California, Estados Unidos, 14 de noviembre, 2018.\\
Luis Jure.

\item[ ]	% ucdavis2018
University of California, Davis, Department of Music.
``Musical Traits and Performance Practice of Uruguayan Candombe Drumming:
A Computational Musicological Approach''.
Davis, California, Estados Unidos, 8 de noviembre, 2018.\\
Luis Jure.

\item[ ]	% columbia2017
Columbia University, Computer Music Center.
``Uruguayan Candombe drumming -- Musical traits and performance practice:
A case study for computational musicology''.
Nueva York, Estados Unidos, 13 de diciembre, 2017.\\
Luis Jure.

\item[ ]	% tu2017
Technische Universität Berlin, Fachgebiet Audiokommunikation.
``Uruguayan Candombe drumming -- Musical traits and performance practice:
A case study for computational musicology''.
Berlín, Alemania, 11 de abril, 2017.\\
Luis Jure.

\item[ ]	% ofai2014
Österreichisches Forschungsinstitut für Artificial Intelligence -- OFAI.
``Tools for detection and classification of piano drum patterns from Candombe recordings''.
Viena, Austria, 12 de diciembre, 2014.\\
Luis Jure.

\item[ ]	% cictem2013
\emph{Congreso Internacional de Ciencia y Tecnología Musical -- CICTeM 2013}.
``Detección automática de patrones rítmicos: el Candombe uruguayo como caso de estudio''.
Buenos Aires, Argentina, 26 al 28 de septiembre, 2013.\\
Luis Jure, Martín Rocamora.

\item[ ]	% ofai2013
Österreichisches Forschungsinstitut für Artificial Intelligence -- OFAI.
``Rhythmic pattern analysis: the Uruguayan Candombe drumming as a case study''
Viena, Austria, 19 de noviembre, 2013.\\
Luis Jure.

\item[ ]	% coloquio2011
Coloquio \emph{La música entre África y América}, Centro Nacional de Documentación
Musical Lauro Ayestarán del Ministerio de Educación y Cultura.
``Principios generativos del toque del repique del Candombe''.
Montevideo, Uruguay, 3 de octubre, 2011.\\
Luis Jure.

\item[ ]	% columbia2010
Columbia University, Coloquio \emph{Sounding the Space}.
``Drums and the City. Musical traits and performance
practices of Candombe in Montevideo''.
Nueva York, Estados Unidos, 8 de octubre, 2010.\\
Luis Jure.

\end{itemize}

